\section{\stech behandler unødig personoplysninger}
Det fremgår, at \stech indsamler og behandler
følgende unødig personoplysninger; kropstemperatur.
I henhold til \gdpr art. 5, stk. 1, litra c, der omtaler dataminimering
er det ikke tilladt at opsamle og behandle data der ikke er
relevante for formålet. Det er ikke nødvendigt at kende til en
persons kropstemperatur, for at kunne informere brugeren stået eller
siddet ned for længe.
Derudover falder kropstemperatur ind under \gdpr art. 4 nr. 14 om
biometrisk data. I henhold til \gdpr art. 9, stk. 1 er det forbudt at
behandle biometrisk data.
\stech' behandling af den unødig personoplysning er i
strid med \gdpr art. 5. og \gdpr art. 9.

\section{\stech videregiver private informationer}
Det fremgår, at \stech' HR-afdeling registrer
følgende følsomme personoplysninger; flytteforhold, adresse,
fagforeningsskift. Disse førnævnte personoplysninger bliver
registreret i mødereferater og videregivet til alle ansatte hos
\stech og
ansatte hos det eksterne system HRSolutions.

I henhold til \gdpr art. 5, stk. 1, litra a, om lovlighed, rimelighed
og gennemsigthed skal behandling af personoplysninger ske rimeligt.
Derudover skal virksomheden \stech sørge for at
formålsbegrænse jf. \gdpr art. 5, stk. 1, litra b.
Videregivelse af information fra
HR-afdelingens møder der vedrører ansatte til alle ansatte i
virksomheden og ansatte i eksterne virksomheder er ikke foreneligt
med \gdpr art. 5, stk. 1, litra a og b.
\gdpr art. 5, stk. 1, litra f om "integritet og fortrolighed", bliver
derved heller ikke overholdt, da \stech videregiver
informationen til ansatte i en ekstern virksomhed. Og kan derved ikke
forsikre sig beskyttelse imod uautoriseret behandling af informationen.
Endvidere er det ikke i strid mod \gdpr art. 9, stk. 1 at \stech
registrer fagforeningsmæssige tilhørsforhold.
\stech' praksis mht. at registrer følsomme personoplysning i
mødereferater og sende disse mødereferater på tværs af virksomheden
og til eksterne virksomheder er derfor i strid med \gdpr art. 5 og 9.

Desuden skal \stech sørge for at overholde \gdpr art. 21 omkring
retten indsigelse eftersom virksomheden er blevet bedt om ikke at
behandle de fornævnte personoplysninger.

\section{Behandling af nødkontakter}
Det fremgår efter en episode, hvor en medarbejde er faldet om, at
virksomheden \stech vil undersøge om det er muligt at have en liste
over nødkontakter med informationer som ansattes familie og deres
kontaktinformationer.

I dette tilfælde, skal \stech være særlig opmærksom på \gdpr art. 6,
stk. 1 om Lovlig behandling.
\stech skal sørge for at have samtykke fra de ansattes nødkontakter,
om at de registrer deres kontaktinformation.
Her skal de også være særlig opmærksom på \gdpr art. 25, stk. 1 \& 2,
kendt som "privacy by design / default".
De skal derfor på forhold forsikre sig at adgangskontrol og
dataminimering er på plads på forhånd, når de implementere den nye
procedure om nødkontakter.

Derudover diskuteres der om der er forskel på behandlingsgrundlaget
for private virksomheder og offentlige myndigheder.
Det er der, offentlige myndigheder kan ikke benytte sig af \gdpr art.
6, stk. 1, litra f, som led i udførelsen af deres opgaver.

\section{\stech distribuerer en ny app, der opsamler oplysninger uden samtykke}
Det fremgår at \stech har udviklet en ny app der tager information
fra deres andet produkt SenseDesk. Appen bliver installeret på alle
arbejdstelefoner via remote installation.

I henhold til \gdpr art. 9, stk. 1 er behandling af personoplysninger
som biometrisk data ikke tilladt.
Behandling af biometrisk data kan kun ske ved undtagelserne fra \gdpr
art. 9, stk. 2, litra a-j. Hvilket \stech ikke har lovhjemmel til.

Derudover siger \gdpr art. 13, stk. 1 at \stech har oplysningspligt
ved indsamling af personoplysninger.
Efter installation af appen på de ansattes arbejdstelefoner, har
\stech ikke givet besked om de relevante oplysninger jf. \gdpr art.
13, stk. 1, litra a-f.

Endvidere, har \stech ikke hjemmel til at installere appen og opsamle
nogen form for
personoplysninger før de har fået eksplicit samtykke fra den
registrerede jf. \gdpr art. 6, stk. 1, litra a.

\stech er heller ikke i overensstemmelse med \gdpr art. 25, stk. 1 \&
2, der siger deres app skal overholde "databeskyttelse gennem design
og standardindstillinger". Appen må derfor ikke ved
standardindstillinger opsamle sunhedsdata fra telefonerne.


Fortsat ses det at Børge bliver modsvaret mht. hans utilfredshed med
den påtvungende installation af appen, at det er en del af deres
ansættelsesforhold.
Dette er i strid imod \dbl § 12. stk. 3 som siger at behandling af
personoplysninger godt kan ske i forbindelse med ansættelsesforhold,
men det skal være i overensstemmelse med \gdpr art. 7.
\gdpr art. 7, nr. 4 siger at samtykket skal være givet frit og at
samtykke ikke giver ret til behandling af personoplysninger som ikke
er nødvendige for opfyldelse af en kontrakt.
\stech kan derfor ikke skjule sig bag § 12, stk. 3 eftersom påtvungen
opsamling af helbredsdata ikke er nødvendig for at kunne opfylde en
ansættelseskontrakt. Samtykket er i denne forbindelse derfor tvunget
og dermed ikke lovligt.

\stech har derfor foretaget ulovlig behandling af personoplysninger,
i strid mod \gdpr art. 6, 9, 13 og 25 og \dbl § 12.

\section{Noget med at \stech ulovlig tvinger deres ansatte til at bruge ansigtsgodkendelse.}
