\section{\stech behandler unødig personoplysninger}
Det fremgår, at \stech indsamler og behandler
følgende unødig personoplysninger; kropstemperatur.
I henhold til \gdpr art. 5, stk. 1, litra c, der omtaler dataminimering
er det ikke tilladt at opsamle og behandle data der ikke er
relevante for formålet. Det er ikke nødvendigt at kende til en
persons kropstemperatur, for at kunne informere brugeren stået eller
siddet ned for længe.
Derudover falder kropstemperatur ind under \gdpr art. 4 nr. 14 om
biometrisk data. I henhold til \gdpr art. 9, stk. 1 er det forbudt at
behandle biometrisk data.
\stech' behandling af den unødig personoplysning er i
strid med \gdpr art. 5. og \gdpr art. 9.

\section{\stech videregiver private informationer}
\stech' HR-afdeling registrerer følsomme personoplysninger som
flytteforhold, adresse og fagforeningsskift. Disse oplysninger
fremgår af mødereferater, som deles med alle ansatte hos \stech samt
hos eksterne HRSolutions.

I henhold til \gdpr art. 5, stk. 1, litra a skal behandling af
personoplysninger ske lovligt, rimeligt og gennemsigtigt.
det fremgår af \gdpr art. 5, stk. 1, litra b
kræver, at oplysninger kun behandles til specifikke, saglige formål.
det fremgår af \gdpr art. 5, stk. 1, litra f fastslår, at oplysninger
skal beskyttes mod uautoriseret
adgang. Registrering af fagforeningsmæssig tilhørsforhold omfattes
ikke af \gdpr art. 9, stk. 1.

\stech' praksis med at registrere følsomme oplysninger i
mødereferater og distribuere dem internt og eksternt strider dermed
mod \gdpr art. 5, stk. 1, litra a,b og f og \gdpr art. 9, stk. 1.
Desuden skal \stech overholde \gdpr art. 21 om retten til indsigelse,
da virksomheden er blevet bedt om ikke at
behandle de nævnte oplysninger.

\section{Behandling af nødkontakter}
Det fremgår efter en episode, hvor en medarbejde er faldet om, at
virksomheden \stech vil undersøge om det er muligt at have en liste
over nødkontakter med informationer som ansattes familie og deres
kontaktinformationer.

I dette tilfælde, skal \stech være særlig opmærksom på \gdpr art. 6,
stk. 1 om Lovlig behandling.
\stech skal sørge for at have samtykke fra de ansattes nødkontakter,
om at de registrer deres kontaktinformation.
Her skal de også være særlig opmærksom på \gdpr art. 25, stk. 1 \& 2,
kendt som "privacy by design / default".
De skal derfor på forhånd forsikre sig at adgangskontrol og
dataminimering er på plads på forhånd, når de implementere den nye
procedure om nødkontakter.

Derudover diskuteres der om der er forskel på behandlingsgrundlaget
for private virksomheder og offentlige myndigheder.
Det er der, offentlige myndigheder kan ikke benytte sig af \gdpr art.
6, stk. 1, litra f, som led i udførelsen af deres opgaver.

\section{\stech distribuerer en ny app, der opsamler oplysninger uden samtykke}
Det fremgår at \stech har udviklet en ny app der tager information
fra deres andet produkt SenseDesk. Appen bliver installeret på alle
arbejdstelefoner via remote installation og indsamler også
helbredsdata fra de ansattes arbejdstelefoner.
De ansatte bliver ikke informeret om installationen og bliver ikke
bedt om samtykke til indsamlingen af den biometriske data.
\stech mener det godt må installere appen på de ansattestelefoner
uden samtykke da det står i ansættelsesforholdet.

I henhold til \gdpr art. 9, stk. 1 er behandling af personoplysninger
som biometrisk data ikke tilladt med undtagelse fra nævnet i \gdpr
art. 9, stk. 2, litra a-j.
Det fremgår af \gdpr art. 13, stk. 1 at den registredede skal
informeres ved indsamling af personoplysninger.
Det fremgår det i \gdpr art. 6, stk. 1 at der kun må behandles
personoplysninger i hvis en af \gdpr art. 6, stk. 1, litra a-f er opfyldt.
Det fremgår i \gdpr art. 25, stk. 1-2 at en virksomhed skal udøve
databestyttelse gennem design og standardindstillinger.
Det fremgår i \dbl § 12, stk. 3 at personoplysninger må behandles
hvis det er i overensstemmelse med \gdpr art. 7.
Det fremgår det i \gdpr art. 7, nr. 4 at samtykke ikke kan
vurderes som frivilligt hvis det er et led i en kontrakt og som ikke
er nødvendig for udførelsen af kontrakten.

\stech er altså i strid imod \gdpr art. 6, stk. 1 og art. 9, stk. 1-2
for at behandle personoplysninger uden samtykke og \gdpr art. 13,
stk. 1 for ikke at informere om denne behandling.
\stech må ikke installere appen ud fra \dbl §12, stk. 3 da det
strider imod \gdpr art. 7, nr. 4, da samtykke til opsamling af
biometrisk data til formålet ikke er nødvendigt for ansættelseskontrakten.
Derudover er \stech i strid imod \gdpr art. 25, stk. 1 da deres
applikation som standard indsamler alle oplysninger om den ansatte.
Der er derfor ikke tale om dataminimering til formålet.

\section{\stech tvinger ansatte til at give biometrisk data}
Firmaet \stech udvikler et system til adgangskontrol til virksomheden
via ansigtgodkendelse.
Det tidligere adgangsmiddel i form af adgangskort, bliver udfaset og
tvinger alle ansatte til at give biometrisk data til ansigtsgodkendelse.
\stech siger at hvis man er uenig i brugen af ansigtsgodkendelse som
det eneste adgangsmiddel, at så må man finde et andet arbejde.

Det er ikke tilladt for \stech at tvinge personalet til at bruge
ansigtsgodkendelse jf. \gdpr art. 9, stk. 1.
Da \stech ikke har en nødvendighed for ansigtsgodkendelse, er det
ikke tilladt for virksomheden at udfase det tidligere adgangsmiddel
og derved tvinge personalet til at give samtykke til
ansigtsgodkendelse, hvilket ikke er tilladt jf. \gdpr betragtning 32,
der siger at samtykke skal være frivilligt.
\stech skal derfor kunne redegøre for at de må behandle biometrisk
data ved at argumentere for \gdpr art. 9, stk. 2, litra a-j. jf.
\gdpr art. 5, stk. 2.

\stech foretager derfor ulovlig behandling af personalets
personoplysninger og påtvinger samtykke, hvilket er i strid mod \gdpr
art. 5 og 9.

\section{For langsom svar og ikke alle informationer i indsigelse}
Det fremgår at der er blevet spurgt om indsigt til personoplysninger
hos \stech, hvor der efter flere måneder er blevet givet et svar som
ikke indeholder alle informationer.
Der fremgår ikke klart om indsigten er blevet sendt via samme system
som den klage, Børge har sendt.

\stech skal jf. \gdpr art. 12, stk. 1 gøre det muligt for indsigt i
alle sine informationer på en
lettilgængelig måde.
Derudover siger \gdpr art. 12, stk. 3 at en indsigelse skal være
besvaret efter max 1 måned. I tilfælde af forsinkelse, må det max
være 2 måneder, hvor i dette tilfælde den registrerede skal
informeres om forsinkelsen.

Eftersom \stech kræver oprettelse af en brugerkonti der kræver
personoplysninger, og først giver svar efter flere måneder uden at
informere om forsinkelsen, og endvidere ikke giver alle informationer
der er blevet efterspurgt, er \stech i strid imod \gdpr art. 12.

\section{Ulovlig videregivelse og brug af adgangsmiddel}
Børge videregiver fortrolig information til uvedkommende i form af et
simpelt password til \stech' centrale system. Bertil deler det med sin
kammerat Kim, og de opnår uberettiget adgang, hvor de bl.a. sletter
Børges personalefil. Kim bruger derefter adgangen til at overføre
20.000 kr. til sin konto i Schweiz og videresælger passwordet på
darkweb.

Ifølge \gdpr art. 32 skal den dataansvarlige sikre passende tekniske
og organisatoriske foranstaltninger. Straffelovens § 263 a, stk. 3,
litra b forbyder videregivelse eller tilegnelse af adgangsmidler til
datasystemer, mens § 263, stk. 1 forbyder uberettiget adgang (hacking).
§ 279a (databedrageri) forbyder uberettiget vinding ved at ændre eller
slette data, og § 290 (hæleri) omfatter besiddelse eller salg af udbytte
fra en strafbar handling.

Børge overtræder § 263 a, stk. 3, ved at dele adgangskoden.
\stech overtræder \gdpr art. 32, da koden ”admin100” er utilstrækkeligt
sikker. Kim og Bertil overtræder § 263 ved at hacke systemet, mens Kim
yderligere overtræder § 279a ved den ulovlige overførsel og § 290 ved
videresalg af adgangskoden.

\section{\stech bruger kode der er beskyttet af ophavsret og vil tage
patent på det}
Det fremgår at 240 linjer af 250 linjers kode i AirAware er identisk
med Martins algoritme til beregning af luftkvalitet.
Det fremgår at de sidste 10 linjer ligner meget Martins algoritme med
ændrede inputvariabler.

Ifølge ophavsretsloven §1, stk. 3, er kode et litterært værk
(herunder open-source kode), og er derfor beskyttet af ophavsret.
Det fremgår af ophavsretsloven § 2, stk. 1-2, at retshaveren har
eneret til hel eller delvis eksemplarfremstilling.
Det fremgår af patentloven § 1, stk. 2, nr. 3 at alene programmer til
datamaskiner ikke kan patenteres.
Det fremgår af patentloven § 2, at en opfindelse skal være ny, have
opfindelseshøjde og industriel anvendelse.

\stech er derfor i strid imod ophavsretsloven §1, stk. 3 og § 2, stk.
1-2, for at benytte sig af Martins kode.
\stech kan endvidere ikke få patent på deres AirAware system da de
ikke har opfindelseshøjde. Det skyldes at der allerede eksistere
lignende systemer,
heriblandt systemet Lindab DCV One.
\stech her ligeledes ikke ret til at patentere kode som de ikke har
rettigheder til. Og ville i denne sammenhæng også skulle fjerne
Martins kode fra systemet.
